\usepackage{contour}
\PassOptionsToPackage{normalem}{ulem}
\usepackage{ulem}
% we do it manually to "fix" it, we do not need breaking
\DeclareRobustCommand*\fancyul[2][\fancyulbackground]{%
   \def\fancyul@background{#1}%
   \def\ULdepth{1pt}%
   \def\ULthickness{.1ex}%
   \contourlength{.35pt}%
   {\color{\fancyulcolor}\uline{\phantom{#2}}}\llap{\contour\fancyul@background{#2}}%
}
\makeatletter
\def\fancyulbackground{btdm@background}%
\colorlet{ul@line@color}{paletteA!70!btdm@background}
\hypersetup{colorlinks=false,pdfborder=0 0 0}% just make sure :D
\def\fancyulcolor{ul@line@color}%
% we make the auto detection of the depth bigger
\def\update@ul{\setbox\z@\hbox{{(j}}\let\ULdepth\p@\edef\ULthickness{\the\dimexpr.235\dp\z@}}
% define a link macro that can be used with href and fancyul
\DeclareRobustCommand*\link{\hyper@normalise\link@do}
\def\link@do#1#2{\leavevmode\hskip-2\p@\href{#1}{\update@ul\hskip2\p@\fancyul{#2}}}% thinspace for some nicer padding
\DeclareRobustCommand*\hlink{\hyper@normalise\hlink@do}
\def\hlink@do#1#2{\leavevmode\hskip-2\p@\hyperlink{#1}{\update@ul\hskip2\p@\fancyul{#2}}}%

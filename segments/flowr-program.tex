\forestset{T/.style={for tree={font=\ttfamily,align=center,l sep=0pt,l sep-=5mm,child anchor=north}}}
\newsavebox\ArchitectureBox
\setbox\ArchitectureBox=\hbox{\tikz[baseline={([yshift=-2mm]current bounding box.center)},scale=1.35]{\draw[rounded corners=.5mm,fill=btdm@background,opacity=.8]
(0,0) -| ++(-6mm,8mm) -- ++(4mm,0) |- ++(2mm,-2mm) coordinate (@) -- cycle
([yshift=-1mm]@) -- ++(0,1mm) -- ++(-2mm,2mm) -- ++(-1mm,0);}}
\newsavebox\RLogo
\setbox\RLogo=\hbox{\tikz[x=.05pt,y=-.05pt]{\draw[even odd rule,fill=btdm@background] (361.453,485.9370) .. controls (162.329,485.9370) and
   (0.906,377.8280) .. (0.906,244.4690) .. controls (0.906,111.1090) and
   (162.329,3.0000) .. (361.453,3.0000) .. controls (560.578,3.0000) and
   (722.,111.1090) .. (722.,244.4690) .. controls (722.,377.8280) and
   (560.578,485.9370) .. (361.453,485.9370) -- cycle(416.641,97.4060) ..
   controls (265.289,97.4060) and (142.594,171.3140) .. (142.594,262.4840) ..
   controls (142.594,353.6540) and (265.289,427.5620) .. (416.641,427.5620) ..
   controls (567.992,427.5620) and (679.687,377.0330) .. (679.687,262.4840) ..
   controls (679.687,147.9710) and (567.992,97.4060) .. (416.641,97.4060) --
   cycle;
\draw[even odd rule,fill=btdm@background,line join=round] (550.,377.0000) .. controls (550.,377.0000) and
   (571.822,383.5850) .. (584.5,390.0000) .. controls (588.899,392.2260) and
   (596.51,396.6680) .. (602.,402.5000) .. controls (607.378,408.2120) and
   (610.,414.0000) .. (610.,414.0000) -- (696.,559.0000) --
   (557.,559.0620) -- (492.,437.0000) .. controls (492.,437.0000) and
   (478.69,414.1310) .. (470.5,407.5000) .. controls (463.668,401.9690) and
   (460.755,400.0000) .. (454.,400.0000) .. controls (449.298,400.0000) and
   (420.974,400.0000) .. (420.974,400.0000) -- (421.,558.9740) --
   (298.,559.0260) -- (298.,152.9380) -- (545.,152.9380) .. controls
   (545.,152.9380) and (657.5,154.9670) .. (657.5,262.0000) .. controls
   (657.5,369.0330) and (550.,377.0000) .. (550.,377.0000) --
   cycle(496.5,241.0240) -- (422.037,240.9760) -- (422.,310.0260) --
   (496.5,310.0020) .. controls (496.5,310.0020) and (531.,309.8950) ..
   (531.,274.8770) .. controls (531.,239.1550) and (496.5,241.0240) ..
   (496.5,241.0240) -- cycle;}%
}
\newsavebox\FirstAst
\setbox\FirstAst=\hbox{\resizebox*!{1.25cm}{\begin{forest}
   for tree={circle,draw,scale=1.65,fill=btdm@background}
   [[[][,phantom]][[[][,phantom]][]]]
\end{forest}}}
\newsavebox\DataFlow
\setbox\DataFlow=\hbox{\resizebox*!{1.25cm}{\color{lightgray!80!black}\begin{forest}
   for tree={circle,draw=black,scale=1.65,fill=btdm@background}
   [[[,name=a][,phantom]][,name=d[[,name=b][,phantom]][,name=c]]]
   \draw[very thick,black,line cap=round] (a) to[bend right=25] (b) (b) to[bend right=25] (c) (c) to[bend right=25] (d);
\end{forest}}}
\newsavebox\Slicing
\setbox\Slicing=\hbox{\resizebox*!{1.25cm}{\color{lightgray!80!black}\begin{forest}
   for tree={circle,draw=black,scale=1.65,fill=btdm@background,opacity=.5}
   [[[,name=a][,phantom]][,name=d[[,name=b,opacity=1,very thick][,phantom]][,name=c,opacity=1,very thick]]]
   \draw[very thick,black,line cap=round,lightgray!80!black] (a) to[bend right=25] (b) (b) to[bend right=25] (c) (c) to[bend right=25] (d);
\end{forest}}}
\newsavebox\Overview
\def\StoreOverview#1{%
\setbox\Overview=\hbox{\begin{tikzpicture}[br/.style={fill=lightgray!25!btdm@background,minimum width=2.5cm,minimum height=1.5cm,signal,signal from=west,signal to=east,signal pointer angle=125},k/.style={below,font=\footnotesize,xshift=-.33mm,color=darkgray},m/.style={above,font=\scriptsize,xshift=-.33mm,color=gray},base/.style={opacity=.35},a/.style={base},b/.style={base},c/.style={base},d/.style={base},e/.style={base},f/.style={base},#1]
   % \scope[transparency group,a]
   % \node[br] (@) at (0,0) {};
   % \node (file) at (@) {\rotatebox{-5}{\usebox\ArchitectureBox}\llap{\T{.r\kern8pt}}};
   % \node[k] at (@.south) {Input File};
   % \endscope

   \scope[transparency group,b]
   \node[br,right=3mm] (@) at (@.east) {};
   \node (r-conv) at (@) {\usebox\RLogo};
   \node[k] at (@.south) {Parsing};
   \endscope

   \scope[transparency group,c]
   \node[br,right=3mm] (@) at (@.east) {};
   \node (first-ast) at (@) {\usebox\FirstAst};
   \node[k] at (@.south) {Normalization};
   \endscope

   \scope[transparency group,d]
   \node[br,right=3mm] (@) at (@.east) {};
   \node (dataflow) at (@) {\usebox\DataFlow};
   \node[k] at (@.south) {Data-Flow};
   \endscope

   \scope[transparency group,e]
   \node[br,right=3mm] (@) at (@.east) {};
   \node (slicing) at (@) {\usebox\Slicing};
   \node[k] at (@.south) {Slicing};
   \endscope

   % \scope[transparency group,f]
   % \node[br,right=3mm] (@) at (@.east) {};
   % \node (export) at (@) {\rotatebox{-5}{\usebox\ArchitectureBox}\llap{\T{.r\kern8pt}}};
   % \node[k] at (@.south) {Output};
   % \endscope
\end{tikzpicture}}}


\section{Current State}
\subsection{The Program}
\begin{frame}{\insertsubsection}
\centering\begin{tikzpicture}[br/.style={fill=lightgray!25!btdm@background,minimum width=2.5cm,minimum height=1.5cm,signal,signal from=west,signal to=east,signal pointer angle=125},k/.style={below,font=\footnotesize,xshift=-.33mm,color=darkgray},m/.style={above,font=\scriptsize,xshift=-.33mm,color=gray},a/.style={}]
   \only<11->{\tikzset{a/.style={opacity=.35}}}
   \begin{uncoverenv}<2->
      \scope[transparency group,a]
      \node[br] (@) at (0,0) {};
      \node (file) at (@) {\rotatebox{-5}{\usebox\ArchitectureBox}\llap{\T{.r\kern8pt}}};
      \node[k] at (@.south) {Input File};
      \endscope
   \end{uncoverenv}

   \begin{uncoverenv}<3->
      \node[br,right=3mm] (@) at (@.east) {};
      \node (r-conv) at (@) {\usebox\RLogo};
      \node[above right=-.5mm,scale=.8,font=\tiny,gray] at(r-conv.south west) {[R]};
      \node[k] at (@.south) {Parsing};
      \onslide<4->{\node[m] at (@.north) {\T{parse}\,\(\to\)\,XML};}
   \end{uncoverenv}

   \begin{uncoverenv}<5->
      \node[br,right=3mm] (@) at (@.east) {};
      \node (first-ast) at (@) {\usebox\FirstAst};
      \node[k] at (@.south) {Normalization};
      \onslide<6->{\node[m] at (@.north) {in TypeScript};}
      \coordinate (@l) at(@.south west);
   \end{uncoverenv}

   \begin{uncoverenv}<7->
      \node[br,right=3mm] (@) at (@.east) {};
      \node (dataflow) at (@) {\usebox\DataFlow};
      \node[k] at (@.south) {Data-Flow};
   \end{uncoverenv}

   \begin{uncoverenv}<8->
      \node[br,right=3mm] (@) at (@.east) {};
      \node (slicing) at (@) {\usebox\Slicing};
      \node[k] at (@.south) {Slicing};
      \onslide<9->{\node[m] at (@.north) {Weiser~\cite{weiser_program_1984}};}
      \coordinate (@r) at(@.south east);
   \end{uncoverenv}

   \begin{uncoverenv}<10->
      \scope[transparency group,a]
      \node[br,right=3mm] (@) at (@.east) {};
      \node (export) at (@) {\rotatebox{-5}{\usebox\ArchitectureBox}\llap{\T{.r\kern8pt}}};
      \node[k] at (@.south) {Output};
      \onslide<9->{\node[m] at (@.north) {Slice,~\ldots};}
      \endscope
   \end{uncoverenv}

   \onslide<11->{\draw[decorate,decoration={brace,mirror}] ([xshift=-1cm,yshift=-6mm]@l) to[edge node={node[below] {\includegraphics[height=2\baselineskip]{logos/flowR.pdf}}}] ([xshift=1mm,yshift=-6mm]@r);}
\end{tikzpicture}
\begin{tikzpicture}[overlay,remember picture]
   \onslide<3->{\node[above left,yshift=\btdmfootheight,gray,font=\tiny] at (current page.south east) {[R]~\url{https://www.r-project.org/logo/}};}
   \onslide<9->{\node[above right,yshift=\btdmfootheight,gray,font=\tiny] at(current page.south west) {\cite{weiser_program_1984}~\citeauthor{weiser_program_1984}, \citetitle{weiser_program_1984} (\citeyear{weiser_program_1984})};}
\end{tikzpicture}
\end{frame}



\StoreOverview{b/.style={},c/.style={}}

\newsavebox\LeftParseTree
\newsavebox\RightParseTree
% \def\Highlight#1{\bfseries\large\color{btdm@primary}#1}

\newsavebox\IfThenElseParseTree
\begin{lrbox}{\IfThenElseParseTree}
\begin{forest}
   T, for tree={l sep=5mm,s sep=5mm}
   [exprlist
      [expr
         [\Content{IF}{if}]
         [\Content{(}{(}]
         [expr
            [\Content{NUM\_CONST}{TRUE}]
         ]
         [\Content{)}{)}]
         [expr
            [\Content{SYMBOL}{x}]
         ]
         [\Content{ELSE}{else}]
         [expr
            [\Content{SYMBOL}{y}]
         ]
      ]
   ]
\end{forest}
\end{lrbox}

\newsavebox\IfThenElseNormalized
\tikzset{K/.style={midway,#1=-.5mm,sloped,font=\footnotesize}}
\begin{lrbox}{\IfThenElseNormalized}
\begin{forest}
   T, for tree={l sep=5mm,s sep=5mm}
   [exprlist
      [\Content{if-then-else}{if (\ldots) \ldots~else \ldots}
         [\Content{boolean}{TRUE (true)}, edge label={node[K=above] {when}}]
         [\Content{symbol}{x}, edge label={node[K=above] {then}}]
         [\Content{symbol}{y}, edge label={node[K=above] {else}}]
      ]
   ]
\end{forest}
\end{lrbox}

\subsection{RQ1: Normalization \& Data-Flow}
\begin{frame}{RQ1: Normalization}
\vspace*{8mm}\begin{center}
\begin{tikzpicture}
   \onslide<2->{\node (@) at (0,0) {%
      \scalebox{.6}{\usebox\IfThenElseParseTree}%
   };
   \node[above] at(@.north) {%
      \footnotesize\bR{parse(text="if (TRUE) x else y")}
   };}

   \onslide<3->{\node[right=3cm] (@2) at (@.east) {\scalebox{.6}{\usebox\IfThenElseNormalized}};

   \draw[-Kite,line cap=round] ([xshift=5mm]@.east|-@2.west) to[edge node={node[above] {\footnotesize\itshape normalized}}] ([xshift=-5mm]@2.west);
   }
\end{tikzpicture}\medskip
\begin{itemize}
   \itemsep3.5pt
   \item<4-> Normalizing constants, namespacing, operators, \ldots
   \item<5-> We use the \link{https://cran.r-project.org/doc/manuals/r-release/R-lang.pdf}{\say{R language definition}}\textsuperscript{\color{gray}\cite{RLang}} as a basis
   % \item<6-> \link{https://roxygen2.r-lib.org/}{Roxygen2}, and comparable tools must be dealt with separately
\end{itemize}
\begin{tikzpicture}[@O]
   \node[FCite] at(current page.south west) {\normalfont\mdseries\cite{RLang} \citeauthor{RLang}, \citetitle{RLang} (\citeyear{RLang})};
\end{tikzpicture}
% \begin{tabular}{p{.4\linewidth}>{\centering\arraybackslash}p{.4\linewidth}}
%    \multicolumn{2}{l}{\onslide<2->{\footnotesize\bR{parse(text="if (TRUE) x else y")}}} \\
%    \onslide<3->{\scalebox{.6}{\usebox\IfThenElseParseTree}} & \onslide<8->{\scalebox{.6}{\usebox\IfThenElseNormalized}} \\
% \end{tabular}
\end{center}%
\begin{tikzpicture}[overlay,remember picture]
   \onslide<1->{\node[below left=5mm,xshift=1mm,scale=.7] at(current page.north east) {\usebox\Overview};}
\end{tikzpicture}%
\note[itemize]{%
\item TODO: Standard ist lange nicht vollständig
}(1)%
\end{frame}
\StoreOverview{d/.style={}}
% \fi


\def\MarkBox#1{#1}%TODO: \fcolorbox{lgray}{btdm@background}{#1}}
\newcommand\MarkAt[3][0]{\setbox0=\hbox{~#3~}\makebox[1mm][c]{\null\hfill\only<#2|handout:#1>{\color{paletteA}\bfseries\fboxsep=1pt\expandafter\MarkBox}\null\hfill}{#3}}
\newsavebox\EmptyEnvBox
\tikzset{%
   Env/.style={text width=1.5cm,rounded corners=2pt,fill=btdm@background},
}%
\def\GetWith#1{\tikz{%
\node[Env] (@) {\scriptsize\strut#1};
\pgfonlayer{background}
\fill[rounded corners=2pt,fill=lgray] ([shift={(1mm,1mm)}]@.north east) rectangle ([shift={(-1mm,-1mm)}]@.south west);
\pgfinterruptboundingbox
\node[above right,gray] at([xshift=-1mm]@.north west) {\scriptsize\strut Environments};
\endpgfinterruptboundingbox
\endpgfonlayer
}}
\newsavebox\EnvBox
\setbox\EnvBox=\hbox{}
\savebox\EmptyEnvBox{\GetWith{}}
\newsavebox\BoxWithX
\savebox\BoxWithX{\GetWith{x}}
\newsavebox\BoxWithr
\savebox\BoxWithr{\GetWith{r}}
\newsavebox\BoxWithrAndX
\savebox\BoxWithrAndX{\GetWith{x, r}}

\newcommand<>\OpaOn[1]{\tikzset{@@/.style={}}\only#2{\tikzset{@@/.style={opacity=.4}}}\scope[transparency group,@@]#1\endscope}
\newsavebox\XDefGraph
\tikzset{
   comm/.style={rectangle,draw=gray,text width=9mm,align=center,minimum height=5mm,font=\ttfamily,fill=lgray},
   d/.style={comm,rounded corners=2pt},
   u/.style={comm,rounded rectangle},
   T/.style={font=\scriptsize,gray}
}
\savebox\XDefGraph{\tikz{%
   \node[d] (@) at (0,0) {x\textsubscript{1}};
\pgfinterruptboundingbox
   \node[above=-1.25mm,gray] at(@.north) {\scriptsize\strut Graph};
\endpgfinterruptboundingbox
}}
\newsavebox\rDefGraph
\savebox\rDefGraph{\tikz{%
   \node[d] (@) at (0,0) {r\textsubscript{1}};
\pgfinterruptboundingbox
   \node[below=-.75mm,gray] at(@.south) {\scriptsize\strut Graph};
\endpgfinterruptboundingbox
}}

\tikzset{K/.style={midway,#1=-.5mm,sloped,font=\smaller[3]}}
\begin{frame}[fragile]{RQ1: Data-Flow}
% TODO: complete from 2.17 with read-graphs propagation
\solsetmintedstyle{plain}
\begin{minted}[escapeinside={/*}{*/}]{R}
/*\onslide<2->*/x <- 1
r <- if(y > 1) x else y/*\onslide<1->*/
\end{minted}
\begin{center}
\begin{onlyenv}<3|handout:1>
\begin{forest}
   T, for tree={l sep=2mm,s sep=5mm,font=\footnotesize}
   [exprlist, s sep=3cm, l sep=0mm
      [\Content{assignment}{\LeftArrow}
         [\Content{symbol}{x}, edge label={node[K=above,sloped] {target}}]
         [\Content{number}{1}, edge label={node[K=above,sloped] {source}}]
      ]
      [\Content{assignment}{\LeftArrow},
         [\Content{symbol}{r}, edge label={node[K=above,sloped] {target}}]
         [\Content{if-then-else}{if (\ldots) \ldots~else \ldots}, edge label={node[K=above,sloped] {source}}
            [\Content{binary-op}{>}\vspace*{-3mm}, edge label={node[K=above,sloped] {when}},
               [\Content{symbol}{y}, edge label={node[K=above,sloped] {lhs}}]
               [\Content{symbol}{1}, edge label={node[K=above,sloped] {rhs}}]
            ]
            [\Content{symbol}{x}, edge label={node[K=above,sloped] {then}}]
            [\Content{symbol}{y}, edge label={node[K=above,sloped] {else}}]
         ]
      ]
   ]
\end{forest}
\end{onlyenv}
\begin{onlyenv}<4-|handout:2>
\begin{forest}
   T, for tree={l sep=2mm,s sep=1.5cm, s sep-=5mm,font=\footnotesize}
   [\MarkAt{5-6}{exprlist},name=exprlist, s sep=3cm, l sep=0mm
      [\MarkAt{7,12}{\LeftArrow},name=la
         [\MarkAt{8-10}{x}, name=x1, edge label={node[K=above,sloped] {target}}]
         [\MarkAt{11}{1}, edge label={node[K=above,sloped] {source}},name=1]
      ]
      [\MarkAt{13}{\LeftArrow},name=la2,s sep=3cm
         [\MarkAt{14}{r}, name=r,edge label={node[K=above,sloped] {target}}]
         [\MarkAt{15,16}{if}, edge label={node[K=above,sloped] {source}},name=if,s sep=1.5cm,
            [>, edge label={node[K=above,sloped] {when}},name=gt,
               [y, edge label={node[K=above,sloped] {lhs}},name=y1]
               [1, edge label={node[K=above,sloped] {rhs}},name=a12]
            ]
            [x, edge label={node[K=above,sloped] {then}},name=x2]
            [y, edge label={node[K=above,sloped] {else}},name=y2]
         ]
      ]
   ]
\pgfinterruptboundingbox
   \onslide<6->{\OpaOn<7->{\node[above right=-1mm,scale=.65] at(exprlist.north east) {\usebox\EmptyEnvBox};}}
   \onslide<7-11|handout:0>{\OpaOn<8->{\node[above left=-1mm,yshift=-1mm,scale=.65] at(la.north west) {\usebox\EmptyEnvBox};}}
   \onslide<8|handout:0>{\node[left,yshift=.5mm,scale=.65] at(x1.west) {\usebox\EmptyEnvBox};}
   \onslide<9->{\OpaOn<10->{\node[left,yshift=.5mm,scale=.65] (@x) at(x1.west) {\usebox\BoxWithX};}}
   \onslide<10->{\OpaOn<11->{\node[below,scale=.65] at(x1.south) {\usebox\XDefGraph};}}
   \onslide<11->{\OpaOn<12->{\node[right=-1mm,scale=.65] at(1.east) {\usebox\EmptyEnvBox};}}
   \onslide<12->
   }
   \onslide<13->{\OpaOn<14->{\node[above right=-1mm,yshift=-1mm,scale=.65] at(la2.north east) {\usebox\EmptyEnvBox};}}
   \onslide<14->{\OpaOn<15->{
      \node[left,scale=.65] at(r.west) {\usebox\BoxWithr};
      \node[below left,scale=.65] at(r.south west) {\usebox\rDefGraph};
   }}
   \onslide<15->{\OpaOn<16->{
      \node[above right,scale=.65] at(if.east) {\usebox\EmptyEnvBox};
   }}
   \onslide<16->{\OpaOn<17->{
      \node[above left,scale=.65] at(gt.north west) {\usebox\EmptyEnvBox};
      \node[left,scale=.65] at (y1.west) {\usebox\EmptyEnvBox};
      \node[below,scale=.65] at(a12.south) {\usebox\EmptyEnvBox};
      \node[below,scale=.65] at(x2.south) {\usebox\EmptyEnvBox};
      \node[below,scale=.65] at(y2.south) {\usebox\EmptyEnvBox};
   }}
\endpgfinterruptboundingbox
\end{forest}
\end{onlyenv}
\end{center}

\begin{tikzpicture}[overlay,remember picture]
   \onslide<1->{\node[below left=5mm,xshift=1mm,scale=.7] at(current page.north east) {\usebox\Overview};}
\end{tikzpicture}%
\note[itemize]{%
\item TODO
}(1)%
\end{frame}

\subsection{RQ2: Common Features}
\begin{frame}[fragile,b]{RQ2: Features}
   \vspace*{9mm}\begin{itemize}[<+(1)->]
      \itemsep\medskipamount
      \item<2-> There are many ways to modify data in R, like: \begin{itemize}
         \item<3-> \bR{a <- 1}, \bR{a <<- 1}, \bR{a = 1}, \bR{1 -> a}, \bR{1 ->> a}
         \item<4-> \bR{assign("a", 1)}, \bR{b <- "a"; assign(b, 1)}
         \item<5-> \bR[morekeywords={[2]{setGeneric}}]{setGeneric("props", function(object) object)}
      \end{itemize}
      \item<6-> Environments and scopes can be changed manually.
%       \solsetmintedstyle{plain}\lstfs{10}%
% \begin{minted}[escapeinside={/*}{*/},aboveskip=0pt]{R}
% /*\onslide<7->*/x <- new.env(); evalq(a <- 1, envir=x); evalq(a, envir=x)/*\onslide<1->*/
% \end{minted}
      \item<8-> Functions can be modified at will (and at any time):
            \solsetmintedstyle{plain}\lstfs{10}%
\begin{minted}[escapeinside={/*}{*/},aboveskip=0pt]{R}
/*\onslide<9->*/f <- function(x) { y <- x * 3; y }
/*\onslide<9->*/body(f)[[3]] <- quote(x)
/*\onslide<9->*/f(2) # 2/*\onslide<1->*/
\end{minted}
      \item<10-> There are different class systems, variable length arguments, and more\ldots % like `...`
   \end{itemize}
\begin{tikzpicture}[overlay,remember picture]
   % \onslide<1->{\node[below left=5mm,xshift=1mm,scale=.7] at(current page.north east) {\usebox\Overview};}
\end{tikzpicture}%
\note[itemize]{%
\item TODO: Package versionen sind unbekannt/nicht angegeben,
}(1)%
\end{frame}

\begin{frame}[fragile]{RQ2: Features\rhead{, II}}
   \begin{itemize}
      \item<2-> \textit{Assumption:} \say{UserRs write different code from package authors.}\bigskip
      \begin{center}
            \begin{tabular}{c@{\hspace*{3.5em}}c}
               \onslide<3->{UseRs} & \onslide<3->{Package Authors} \medskip\\
               \onslide<4->{published scripts in social science} & \onslide<5->{top 500 CRAN packages} \\
               \onslide<6->{\num{4230} files} & \onslide<7->{\num{25\,691} files}
            \end{tabular}
      \end{center}
   \end{itemize}%
\note[itemize]{%
\item TODO
}(1)%
\end{frame}

\begin{frame}[c]{RQ2: Features\rhead{, III}}
   \centering
   \onslide<2->{\includegraphics{statistics/stat-assignments-I-.pdf}}%
   \hfill\onslide<3->{\includegraphics{statistics/stat-assignments-II-.pdf}}%
   \hfill\onslide<4->{\includegraphics{statistics/stat-assignments-=.pdf}}%
   %\bigskip\\%
   % ~\onslide<5->{\includegraphics{statistics/stat-assignments--I.pdf}}%
   % ~\onslide<6->{\includegraphics{statistics/stat-assignments--II.pdf}}%
   % ~\onslide<7->{\includegraphics{statistics/stat-assignments-:=.pdf}}
\end{frame}

\begin{frame}[c]{RQ2: Features\rhead{, IV}}
   \centering
   \begin{itemize}
      \item Meistgenutzte Pakete \(\implies\) Spezielle Unterstützung [TODO: hier nur noch Zahlen nicht pro datei etc.]
      \item Verwendung von assign/setGeneric zu häufig \(\implies\) Spezielle Unterstützung [TODO: hier nochmal eine Grafik für gruppiert assignNamespace etc.]
      \item Unterstztüzung für Datentypen usw. erstmal nur Zugriff per Name (meistgenutzt)
      \item .C, .Fortran etc. konservativ abschätzen
   \end{itemize}

   % TODO: 2-3 interesting findings (meistgenutzte Pakete,~\ldots)
\end{frame}

\begin{frame}[c]{RQ3: TODO [Slicing Block rechts oben]}
   \centering
   TODO: kurz auf die gezeigten features eingehen.
   Scopes machen wir indem jede Funktion erstmal einen Datenflussgraph kriegt, der dann bei jedem Aufruf neu gebunden werden kann. ebenso sind environment informationen teil dees datenflussgraphen
   sowie: unsicherheit...
\end{frame}

% \begin{frame}[c]{RQ4: TODO}
%    \centering
%    Steht noch aus.
% \end{frame}

\section{Outlook}
\begin{frame}{Kurze übersicht darüber, was noch gemacht werden muss}
   [Zeitplan,~\ldots]
\end{frame}
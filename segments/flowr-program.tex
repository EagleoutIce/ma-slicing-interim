\newsavebox\ArchitectureBox
\setbox\ArchitectureBox=\hbox{\tikz[baseline={([yshift=-2mm]current bounding box.center)},scale=1.35]{\draw[rounded corners=.5mm,fill=btdm@background,opacity=.8]
(0,0) -| ++(-6mm,8mm) -- ++(4mm,0) |- ++(2mm,-2mm) coordinate (@) -- cycle
([yshift=-1mm]@) -- ++(0,1mm) -- ++(-2mm,2mm) -- ++(-1mm,0);}}
\newsavebox\RLogo
\setbox\RLogo=\hbox{\tikz[x=.05pt,y=-.05pt]{\draw[even odd rule,fill=btdm@background] (361.453,485.9370) .. controls (162.329,485.9370) and
   (0.906,377.8280) .. (0.906,244.4690) .. controls (0.906,111.1090) and
   (162.329,3.0000) .. (361.453,3.0000) .. controls (560.578,3.0000) and
   (722.,111.1090) .. (722.,244.4690) .. controls (722.,377.8280) and
   (560.578,485.9370) .. (361.453,485.9370) -- cycle(416.641,97.4060) ..
   controls (265.289,97.4060) and (142.594,171.3140) .. (142.594,262.4840) ..
   controls (142.594,353.6540) and (265.289,427.5620) .. (416.641,427.5620) ..
   controls (567.992,427.5620) and (679.687,377.0330) .. (679.687,262.4840) ..
   controls (679.687,147.9710) and (567.992,97.4060) .. (416.641,97.4060) --
   cycle;
\draw[even odd rule,fill=btdm@background,line join=round] (550.,377.0000) .. controls (550.,377.0000) and
   (571.822,383.5850) .. (584.5,390.0000) .. controls (588.899,392.2260) and
   (596.51,396.6680) .. (602.,402.5000) .. controls (607.378,408.2120) and
   (610.,414.0000) .. (610.,414.0000) -- (696.,559.0000) --
   (557.,559.0620) -- (492.,437.0000) .. controls (492.,437.0000) and
   (478.69,414.1310) .. (470.5,407.5000) .. controls (463.668,401.9690) and
   (460.755,400.0000) .. (454.,400.0000) .. controls (449.298,400.0000) and
   (420.974,400.0000) .. (420.974,400.0000) -- (421.,558.9740) --
   (298.,559.0260) -- (298.,152.9380) -- (545.,152.9380) .. controls
   (545.,152.9380) and (657.5,154.9670) .. (657.5,262.0000) .. controls
   (657.5,369.0330) and (550.,377.0000) .. (550.,377.0000) --
   cycle(496.5,241.0240) -- (422.037,240.9760) -- (422.,310.0260) --
   (496.5,310.0020) .. controls (496.5,310.0020) and (531.,309.8950) ..
   (531.,274.8770) .. controls (531.,239.1550) and (496.5,241.0240) ..
   (496.5,241.0240) -- cycle;}%
}
\newsavebox\FirstAst
\setbox\FirstAst=\hbox{\resizebox*!{1.25cm}{\begin{forest}
   for tree={circle,draw,scale=1.65,fill=btdm@background}
   [[[][,phantom]][[[][,phantom]][]]]
\end{forest}}}
\newsavebox\DataFlow
\setbox\DataFlow=\hbox{\resizebox*!{1.25cm}{\color{lightgray!80!black}\begin{forest}
   for tree={circle,draw=black,scale=1.65,fill=btdm@background}
   [[[,name=a][,phantom]][,name=d[[,name=b][,phantom]][,name=c]]]
   \draw[very thick,black,line cap=round] (a) to[bend right=25] (b) (b) to[bend right=25] (c) (c) to[bend right=25] (d);
\end{forest}}}
\newsavebox\Slicing
\setbox\Slicing=\hbox{\resizebox*!{1.25cm}{\color{lightgray!80!black}\begin{forest}
   for tree={circle,draw=black,scale=1.65,fill=btdm@background,opacity=.5}
   [[[,name=a][,phantom]][,name=d[[,name=b,opacity=1,very thick][,phantom]][,name=c,opacity=1,very thick]]]
   \draw[very thick,black,line cap=round,lightgray!80!black] (a) to[bend right=25] (b) (b) to[bend right=25] (c) (c) to[bend right=25] (d);
\end{forest}}}
\newsavebox\Overview
\def\StoreOverview#1{%
\setbox\Overview=\hbox{\begin{tikzpicture}[br/.style={fill=lightgray!25!btdm@background,minimum width=2.5cm,minimum height=1.5cm,signal,signal from=west,signal to=east,signal pointer angle=125},k/.style={below,font=\footnotesize,xshift=-.33mm,color=darkgray},m/.style={above,font=\scriptsize,xshift=-.33mm,color=gray},base/.style={opacity=.35},a/.style={base},b/.style={base},c/.style={base},d/.style={base},e/.style={base},f/.style={base},#1]
   \scope[transparency group,a]
   \node[br] (@) at (0,0) {};
   \node (file) at (@) {\rotatebox{-5}{\usebox\ArchitectureBox}\llap{\T{.r\kern8pt}}};
   \node[k] at (@.south) {Input File};
   \endscope

   \scope[transparency group,b]
   \node[br,right=3mm] (@) at (@.east) {};
   \node (r-conv) at (@) {\usebox\RLogo};
   \node[k] at (@.south) {Parsing};
   \endscope

   \scope[transparency group,c]
   \node[br,right=3mm] (@) at (@.east) {};
   \node (first-ast) at (@) {\usebox\FirstAst};
   \node[k] at (@.south) {Normalization};
   \endscope

   \scope[transparency group,d]
   \node[br,right=3mm] (@) at (@.east) {};
   \node (dataflow) at (@) {\usebox\DataFlow};
   \node[k] at (@.south) {Dataflow};
   \endscope

   \scope[transparency group,e]
   \node[br,right=3mm] (@) at (@.east) {};
   \node (slicing) at (@) {\usebox\Slicing};
   \node[k] at (@.south) {Slicing};
   \endscope

   \scope[transparency group,f]
   \node[br,right=3mm] (@) at (@.east) {};
   \node (export) at (@) {\rotatebox{-5}{\usebox\ArchitectureBox}\llap{\T{.r\kern8pt}}};
   \node[k] at (@.south) {Output};
   \endscope
\end{tikzpicture}}}


\section{First Results}
\subsection{The Program}
\begin{frame}{\insertsubsection}
\centering\begin{tikzpicture}[br/.style={fill=lightgray!25!btdm@background,minimum width=2.5cm,minimum height=1.5cm,signal,signal from=west,signal to=east,signal pointer angle=125},k/.style={below,font=\footnotesize,xshift=-.33mm,color=darkgray},m/.style={above,font=\scriptsize,xshift=-.33mm,color=gray}]
   \begin{uncoverenv}<2->
      \node[br] (@) at (0,0) {};
      \node (file) at (@) {\rotatebox{-5}{\usebox\ArchitectureBox}\llap{\T{.r\kern8pt}}};
      \node[k] at (@.south) {Input File};
   \end{uncoverenv}

   \begin{uncoverenv}<3->
      \node[br,right=3mm] (@) at (@.east) {};
      \node (r-conv) at (@) {\usebox\RLogo};
      \node[above right=-.5mm,scale=.8,font=\tiny,gray] at(r-conv.south west) {[R]};
      \node[k] at (@.south) {Parsing};
      \onslide<4->{\node[m] at (@.north) {\T{parse}\,\(\to\)\,XML};}
   \end{uncoverenv}

   \begin{uncoverenv}<5->
      \node[br,right=3mm] (@) at (@.east) {};
      \node (first-ast) at (@) {\usebox\FirstAst};
      \node[k] at (@.south) {Normalization};
      \onslide<6->{\node[m] at (@.north) {in TypeScript};}
   \end{uncoverenv}

   \begin{uncoverenv}<7->
      \node[br,right=3mm] (@) at (@.east) {};
      \node (dataflow) at (@) {\usebox\DataFlow};
      \node[k] at (@.south) {Data-Flow};
   \end{uncoverenv}

   \begin{uncoverenv}<8->
      \node[br,right=3mm] (@) at (@.east) {};
      \node (slicing) at (@) {\usebox\Slicing};
      \node[k] at (@.south) {Slicing};
      \onslide<9->{\node[m] at (@.north) {Weiser~\cite{weiser_program_1984}};}
   \end{uncoverenv}

   \begin{uncoverenv}<10->
      \node[br,right=3mm] (@) at (@.east) {};
      \node (export) at (@) {\rotatebox{-5}{\usebox\ArchitectureBox}\llap{\T{.r\kern8pt}}};
      \node[k] at (@.south) {Output};
      \onslide<9->{\node[m] at (@.north) {Slice,~\ldots};}
   \end{uncoverenv}
\end{tikzpicture}
\begin{tikzpicture}[overlay,remember picture]
   \onslide<3->{\node[above left,yshift=\btdmfootheight,gray,font=\tiny] at (current page.south east) {[R]~\url{https://www.r-project.org/logo/}};}
   \onslide<9->{\node[above right,yshift=\btdmfootheight,gray,font=\tiny] at(current page.south west) {\cite{weiser_program_1984}~\citeauthor{weiser_program_1984}, \citetitle{weiser_program_1984} (\citeyear{weiser_program_1984})};}
\end{tikzpicture}
\end{frame}



\StoreOverview{b/.style={},c/.style={}}

\newsavebox\LeftParseTree
\newsavebox\RightParseTree

\forestset{T/.style={for tree={font=\ttfamily,align=center,l sep=0pt,l sep-=5mm,child anchor=north}}}
% \def\Highlight#1{\bfseries\large\color{btdm@primary}#1}

\newsavebox\IfThenElseParseTree
\begin{lrbox}{\IfThenElseParseTree}
\begin{forest}
   T, for tree={l sep=5mm,s sep=5mm}
   [exprlist
      [expr
         [\Content{IF}{if}]
         [\Content{(}{(}]
         [expr
            [\Content{NUM\_CONST}{TRUE}]
         ]
         [\Content{)}{)}]
         [expr
            [\Content{SYMBOL}{x}]
         ]
         [\Content{ELSE}{else}]
         [expr
            [\Content{SYMBOL}{y}]
         ]
      ]
   ]
\end{forest}
\end{lrbox}

\newsavebox\IfThenElseNormalized
\tikzset{K/.style={midway,#1=3mm,font=\footnotesize}}
\begin{lrbox}{\IfThenElseNormalized}
\begin{forest}
   T, for tree={l sep=5mm,s sep=5mm}
   [exprlist
      [\Content{if-then-else}{if (\ldots) \ldots~else \ldots}
         [\Content{boolean}{TRUE (true)}, edge label={node[K=left] {condition}}]
         [\Content{symbol}{x}, edge label={node[K=right] {then}}]
         [\Content{symbol}{y}, edge label={node[K=right] {otherwise}}]
      ]
   ]
\end{forest}
\end{lrbox}

\subsection{RQ1: Normalization \& Data-Flow}
\begin{frame}{RQ1: Normalization}
\vspace*{8mm}\begin{center}
\begin{tikzpicture}
   \onslide<2->{\node (@) at (0,0) {%
      \scalebox{.6}{\usebox\IfThenElseParseTree}%
   };
   \node[above] at(@.north) {%
      \footnotesize\bR{parse(text="if (TRUE) x else y")}
   };}

   \onslide<3->{\node[right=3cm] (@2) at (@.east) {\scalebox{.6}{\usebox\IfThenElseNormalized}};

   \draw[-Kite,line cap=round] ([xshift=5mm]@.east|-@2.west) to[edge node={node[above] {\footnotesize\itshape normalized}}] ([xshift=-5mm]@2.west);
   }
\end{tikzpicture}\medskip
\begin{itemize}
   \itemsep3.5pt
   \item<4-> Normalizing constants, namespacing, operators, \ldots
   \item<5-> We use the \link{https://cran.r-project.org/doc/manuals/r-release/R-lang.pdf}{\say{R language definition}}\textsuperscript{\color{gray}\cite{RLang}} as a basis
   % \item<6-> \link{https://roxygen2.r-lib.org/}{Roxygen2}, and comparable tools must be dealt with separately
\end{itemize}
\citeon<5->{RLang}
% \begin{tabular}{p{.4\linewidth}>{\centering\arraybackslash}p{.4\linewidth}}
%    \multicolumn{2}{l}{\onslide<2->{\footnotesize\bR{parse(text="if (TRUE) x else y")}}} \\
%    \onslide<3->{\scalebox{.6}{\usebox\IfThenElseParseTree}} & \onslide<8->{\scalebox{.6}{\usebox\IfThenElseNormalized}} \\
% \end{tabular}
\end{center}%
\begin{tikzpicture}[overlay,remember picture]
   \onslide<1->{\node[below left=5mm,xshift=1mm,scale=.7] at(current page.north east) {\usebox\Overview};}
\end{tikzpicture}%
\note[itemize]{%
\item TODO: Standard ist lange nicht vollständig
}(1)%
\end{frame}



\StoreOverview{d/.style={}}
\begin{frame}{RQ1: Data-Flow}
TODO: wie viel?
\begin{tikzpicture}[overlay,remember picture]
   \onslide<1->{\node[below left=5mm,xshift=1mm,scale=.7] at(current page.north east) {\usebox\Overview};}
\end{tikzpicture}%
\note[itemize]{%
\item TODO
}(1)%
\end{frame}

\subsection{RQ2: Common Features}
\begin{frame}[fragile,b]{RQ2: Features}
   \vspace*{9mm}\begin{itemize}[<+(1)->]
      \itemsep\medskipamount
      \item<2-> There are many ways to modify data in R, like: \begin{itemize}
         \item<3-> \bR{a <- 1}, \bR{a <<- 1}, \bR{a = 1}, \bR{1 -> a}, \bR{1 ->> a}
         \item<4-> \bR{assign("a", 1)}, \bR{b <- "a"; assign(b, 1)}
         \item<5-> \bR[morekeywords={[2]{setGeneric}}]{setGeneric("props", function(object) object)}
      \end{itemize}
      \item<6-> Environments and scopes can be changed manually.
%       \solsetmintedstyle{plain}\lstfs{10}%
% \begin{minted}[escapeinside={/*}{*/},aboveskip=0pt]{R}
% /*\onslide<7->*/x <- new.env(); evalq(a <- 1, envir=x); evalq(a, envir=x)/*\onslide<1->*/
% \end{minted}
      \item<8-> Functions can be modified at will (and at any time):
            \solsetmintedstyle{plain}\lstfs{10}%
\begin{minted}[escapeinside={/*}{*/},aboveskip=0pt]{R}
/*\onslide<9->*/f <- function(x) { y <- x * 3; y }
/*\onslide<9->*/body(f)[[3]] <- quote(x)
/*\onslide<9->*/f(2) # 2/*\onslide<1->*/
\end{minted}
      \item<10-> There are different class systems, variable length arguments, and more\ldots % like `...`
   \end{itemize}
\begin{tikzpicture}[overlay,remember picture]
   \onslide<1->{\node[below left=5mm,xshift=1mm,scale=.7] at(current page.north east) {\usebox\Overview};}
\end{tikzpicture}%
\note[itemize]{%
\item TODO: Package versionen sind unbekannt/nicht angegeben,
}(1)%
\end{frame}

\begin{frame}[fragile]{RQ2: Features\rhead{, II}}
   \begin{itemize}
      \item<2-> \textit{Assumption:} \say{UserRs write different code from package authors.}\bigskip
      \begin{center}
            \begin{tabular}{c@{\hspace*{3.5em}}c}
               \onslide<3->{UseRs} & \onslide<3->{Package Authors} \medskip\\
               \onslide<4->{published sources in social science} & \onslide<5->{top 500 CRAN packages} \\
               \onslide<6->{\num{4230} files} & \onslide<7->{\num{25\,691} files}
            \end{tabular}
      \end{center}
   \end{itemize}%
\note[itemize]{%
\item TODO
}(1)%
\end{frame}

\begin{frame}[c]{RQ2: Features\rhead{, III}}
   \centering
   \onslide<2->{\includegraphics{statistics/stat-example.pdf}} [TODO: andere]
\end{frame}

\begin{frame}[c]{RQ2: Features\rhead{, IV}}
   \centering
   TODO: 2-3 interesting findings (meistgenutzte pakete,~\ldots)
\end{frame}

\begin{frame}[c]{RQ3: TODO}
   \centering
   TODO: kurz auf die gezeigten features eingehen.
   Scopes machen wir indem jede Funktion erstmal einen Datenflussgraph kriegt, der dann bei jedem Aufruf neu gebunden werden kann. ebenso sind environment informationen teil dees datenflussgraphen
   sowie: unsicherheit...
\end{frame}

\begin{frame}[c]{RQ4. TODO}
   \centering
   Steht noch aus.
\end{frame}

\section{Outlook}
\begin{frame}{Kurze übersicht darüber, was noch gemacht werden muss}
   [Zeitplan,~\ldots]
\end{frame}
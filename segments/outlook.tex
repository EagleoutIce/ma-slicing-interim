
\section{Outlook}
\def\W{.85}
% color | row | from | to
\def\Data[#1]#2#3#4{
    \filldraw[#1!75!white,rounded corners=1.25pt] (#2.east)++(2mm+#3*1.2cm,-4.5pt) rectangle ++(#4*1.2cm,9pt);
}
\begin{frame}{The Plan}
\vspace*{-6mm}% completed: 80 days future: 106 days
\frametitle<-4|handout:1>{The Plan}
\frametitle<5-|handout:2>{The Reality}
\begin{onlyenv}<-4|handout:1>
\vspace*{-2mm}\only<2->{\resizebox{.9\linewidth}!{\begin{tikzpicture}[t/.style={gray},xscale=1.15]
  \foreach[count=\i] \k in {Writing,Research,Static,Feature,Pointer,Evaluation,Finalize} {
    \colorlet{@}{gray}
    \only<4->{\colorlet{@}{hlgray!62!white}}
    \ifnum\i=1\colorlet{@}{gray}\fi\ifnum\i=5\colorlet{@}{gray}\fi
    \onslide<2->{\node[t,@,left] (\k) at (0,-\i*5.5mm) {\k};}
    % guide lines:
    \pgfonlayer{background}
    \onslide<2->{\draw[line cap=round,line width=1.33pt,dfgray!14!white] (\k.east)++(1mm,0) -- ++(6*1.2cm+2mm,0);}
    \endpgfonlayer
  }
  \onslide<3->{
    \colorlet{@}{gray!80!white}
  \only<4->{\colorlet{@}{gray!45!white}}
  \Data[gray!80!white]{Writing}{0}{6}
  \Data[@]{Research}{0}{.5}
  \Data[@]{Static}{.5}{1} % \Data[@]{Structure}{2}{.5} \Data[@]{Structure}{3}{.5}
  \Data[@]{Feature}{1.25}{.75}
  \Data[gray!80!white]{Pointer}{2}{2.5} % \Data[@]{Verification}{4}{2}
  \Data[@]{Evaluation}{4.5}{0.5}
  \Data[@]{Finalize}{5}{1}}
  \foreach\i in {1,...,7} {
      \ifnum\i<7
        \node[t,above] at (-1cm+\i*1.2cm+.6cm,0) {\only<4->{\ifnum\i=3\bfseries\else\color{hlgray}\fi}M\i};
      \fi
      % opacity away from 4 => lower
      \pgfonlayer{background}
      \draw[hlgray,opacity=.33,line width=1.33pt,line cap=round] (-1cm+\i*1.2cm,.5cm) -- ++(0,-4.75cm);
      \endpgfonlayer
  }
  \onslide<4->{
  \draw[paletteA,opacity=.8,line width=2pt,line cap=round] (6*1.2cm*0.430108,0) -- ++(0,-4.5cm) node[above right,outer sep=0pt,inner sep=0pt] (20) {\tiny\bfseries~~24.};
  \node[below right,shadeA, outer sep=0pt, inner sep=0pt,yshift=-2pt] at(20.south west) {\tiny\bfseries~~May};
  }
\end{tikzpicture}}}
\end{onlyenv}
\begin{onlyenv}<5-|handout:2>
\vspace*{-2mm}\resizebox{.9\linewidth}!{\begin{tikzpicture}[t/.style={gray},xscale=1.15,yscale=.85]
  \foreach[count=\i] \k in {Writing,Research,Normalize,Data-Flow,Static,Feature,Pointer,Evaluation,Finalize} {
    \colorlet{@}{gray}
    \colorlet{@}{hlgray!62!white}
    \ifnum\i=1\colorlet{@}{gray}\fi\ifnum\i=3\colorlet{@}{gray}\fi\ifnum\i=4\colorlet{@}{gray}\fi\ifnum\i=7\colorlet{@}{gray}\fi
    \node[t,@,left] (\k) at (0,-\i*5.5mm) {\k};
    % guide lines:
    \pgfonlayer{background}
    \draw[line cap=round,line width=1.33pt,dfgray!14!white] (\k.east)++(1mm,0) -- ++(6*1.2cm+2mm,0);
    \endpgfonlayer
  }
  \colorlet{@}{gray!80!white}
  \colorlet{@}{gray!45!white}
  \Data[gray!80!white]{Writing}{0}{6}
  \Data[@]{Research}{0}{.5}
  \Data[gray!80!white]{Normalize}{0.75}{2.5}
  \Data[gray!80!white]{Data-Flow}{1}{3.5}
  \Data[@]{Static}{.5}{1} % \Data[@]{Structure}{2}{.5} \Data[@]{Structure}{3}{.5}
  \Data[@]{Feature}{1.25}{.75}
  \Data[gray!80!white]{Pointer}{2}{2.5} % \Data[@]{Verification}{4}{2}
  \Data[@]{Evaluation}{4.5}{0.5}
  \Data[@]{Finalize}{5}{1}
  \foreach\i in {1,...,7} {
      \ifnum\i<7
        \node[t,above] at (-1cm+\i*1.2cm+.6cm,0) {\ifnum\i=3\bfseries\else\color{hlgray}\fi M\i};
      \fi
      % opacity away from 4 => lower
      \pgfonlayer{background}
      \draw[hlgray,opacity=.33,line width=1.33pt,line cap=round] (-1cm+\i*1.2cm,.5cm) -- ++(0,-6cm);
      \endpgfonlayer
  }
  \draw[paletteA,opacity=.8,line width=2pt,line cap=round] (6*1.2cm*0.430108,0) -- ++(0,-5.75cm) node[above right,outer sep=0pt,inner sep=0pt] (20) {\tiny\bfseries~~24.};
  \node[below right,shadeA, outer sep=0pt, inner sep=0pt,yshift=-2pt] at(20.south west) {\tiny\bfseries~~May};
\end{tikzpicture}}
\end{onlyenv}
\note[itemize]{%
\item Doch wo stehe ich jetzt gerade?
\item Das war der Plan
\item Realität: Normalisierung und Datenflussanalyse komemn dazu
\item Aktive Phasen: \textbf{Normalize/DF} und \textbf{Pointer}
\item Pointer ist trennen von Array-Komponenten/Data-Frames/in R6 Reference classes schauen etc.
}(1)%%
\end{frame}

\tikzset{
    path image shift/.style={},
    path image/.style={path picture={\node at ([path image shift]path picture bounding box.center) {#1};}}
}
\newsavebox\OurGoalBox \savebox\OurGoalBox{~~~~\resizebox{6cm}!{\usebox\FinalSlicingStage}~}
\StoreOverview{a/.style={},b/.style={},c/.style={},d/.style={},e/.style={},f/.style={}}
\newsavebox\OurOverviewBox \savebox\OurOverviewBox{~~~\resizebox{6cm}!{\usebox\Overview}~~~}
\newsavebox\OurDataflow \savebox\OurDataflow{~~~\resizebox{4.25cm}!{\usebox\FinalDataFlow}~~~}
\newsavebox\OurFeature \savebox\OurFeature{~~~\resizebox{5cm}!{\includegraphics{statistics/stat-assignments-I-.pdf}%
\qquad\includegraphics{statistics/stat-assignments-II-.pdf}%
\qquad\includegraphics{statistics/stat-assignments-=.pdf}}~~~}
\begin{frame}{Overview}
\begin{tikzpicture}
  \onslide<2->{\draw[path image={\hyperlink{@Goal}{\usebox\OurGoalBox}},rounded corners=2pt,draw] (0,0) rectangle ++(\wd\OurGoalBox,\ht\OurGoalBox+\dp\OurGoalBox);
  \node[below] at(0.5\wd\OurGoalBox,0) {\textbf{Goal}};}

  \onslide<3->{\draw[path image={\hyperlink{@Program}{\usebox\OurOverviewBox}},rounded corners=2pt,draw] (\wd\OurGoalBox+1cm,0) rectangle ++(\wd\OurOverviewBox,\ht\OurGoalBox+\dp\OurGoalBox);
  \node[below] at(\wd\OurGoalBox+1cm+0.5\wd\OurOverviewBox,0) {\textbf{Program}};}

  \onslide<4->{\draw[path image={\hyperlink{@DataFlow}{\usebox\OurDataflow}},rounded corners=2pt,draw] (0,-\ht\OurGoalBox-\dp\OurGoalBox-1cm) rectangle ++(\wd\OurGoalBox,\ht\OurGoalBox+\dp\OurGoalBox);
  \node[below] at(0.5\wd\OurGoalBox,-\ht\OurGoalBox-\dp\OurGoalBox-1cm) {\textbf{Data-Flow}};}

  \onslide<5->{\draw[path image={\hyperlink{@Feature}{\usebox\OurFeature}},rounded corners=2pt,draw] (\wd\OurGoalBox+1cm,-\ht\OurGoalBox-\dp\OurGoalBox-1cm) rectangle ++(\wd\OurOverviewBox,\ht\OurGoalBox+\dp\OurGoalBox);
  \node[below] at(\wd\OurGoalBox+1cm+0.5\wd\OurOverviewBox,-\ht\OurGoalBox-\dp\OurGoalBox-1cm) {\textbf{Features}};}
\end{tikzpicture}
\note[itemize]{%
\item Das Ziel und was wir dafür brauchen (AST, DF, Types, Slicing Algorihtmus)
\item flowR + Phasen
\item Besonders: Datenfluss
\item Features und was wir daraus lernen können
\item **Überleitung Bibliographie und Fragen**
}(1)%%
\end{frame}
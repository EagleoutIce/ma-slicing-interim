
\section{Goal}
\subsection{Program Slicing}
\newsavebox\FinalSlicingStage
\begin{frame}[fragile]{The Goal}
\vspace*{-7.5mm}\lstfs{9}
\null\hfill\begin{lrbox}\FinalSlicingStage
\begin{tikzpicture}
\begin{uncoverenv}<2->
\node (@) at (0,0) {%
\begin{minted}[deletekeywords={sum,prod},escapeinside={/*}{*/}]{R}
sum  <- 0
prod <- 1
n    <- 10

for (i in 1:(n-1)) {
   sum  <- sum + i
   prod <- prod * i
}

cat("Sum:", /*\T{\only<3->{\textbf}{sum}}*/, "\n")
cat("Product:", prod, "\n")
\end{minted}
};
\end{uncoverenv}

\begin{uncoverenv}<4->
   \node[right=2.5cm] (@2) at (@.east) {%
\solsetmintedstyle{plain}
\AnimateCode{onslide={o8:{7,5,4,3,1}}, first slide=4,handout={4/1}}
\begin{minted}[deletekeywords={sum,prod},escapeinside={/*}{*/}]{R}
sum  <- 0
prod <- 1
n    <- 10

for (i in 1:(n-1)) {
   sum  <- sum + i
   prod <- prod * i
}

cat("Sum:", sum, "\n")
cat("Product:", prod, "\n")
\end{minted}
\endAnimateCode
   };
\end{uncoverenv}
\onslide<3->{
   \draw[-Kite,line cap=round] ([xshift=5mm]@.east) to[edge node={node[above] {\footnotesize\T{slice(\solGet{linenumbers}{\footnotesize10}, \textbf{sum})}}}] ([xshift=-5mm]@2.west);
}
\end{tikzpicture}\end{lrbox}\usebox\FinalSlicingStage\hfill\null
\global\setbox\FinalSlicingStage=\box\FinalSlicingStage
\note[itemize]{%
\item TODO
}(1)%
\end{frame}

\def\Content#1#2{#1\\[-.8ex]\small\textit{#2}}
\newsavebox\ParseTree

\begin{lrbox}{\ParseTree}
\begin{forest}
   for tree={font=\ttfamily,align=center,l sep=0pt,l sep-=5mm}
   [exprlist
      [expr
         [expr
            [\Content{SYMBOL}{x}]
         ]
         [\Content{LEFT\_ASSIGN}{{\LeftArrow/\DoubleLeftArrow}}]
         [expr
            [expr
               [\Content{NUM\_CONST}{4}]
            ]
            [\Content{\smash{\textasciicircum}}{{\textasciicircum/**}}]
            [expr
               [\Content{NUM\_CONST}{3}]
            ]
         ]
      ]
   ]
\end{forest}
\end{lrbox}

\def\After#1{\quad\textcolor{gray}{\scriptsize#1}}
\subsection{Program Slicing Requirements}
\begin{frame}[c]{Requirements [TODO: more related on other slicing algorithms?]}
\vspace*{\smallskip}\begin{enumerate}
   \itemsep12pt
   \item<2-> Control-flow information (AST) \onslide<11-|handout:2->{\After{partially}}\begin{onlyenv}<6-10|handout:1>
         \begin{itemize}
         \item<6-> R provides a \T{parse} function to parse R code
         \item<7-> But the produced AST is inconsistent
      \end{itemize}
   \end{onlyenv}
   \item<3-> Data-flow information \onslide<14-|handout:3->{\After{nothing}}\begin{onlyenv}<12-13|handout:2>
      \begin{itemize}
      \item<12-> There is \T{CodeDepends} which does not differentiate bodies
      \item<13-> Otherwise: No existing data-flow analysis.
   \end{itemize}
\end{onlyenv}
   \item<4-> Type information \onslide<17-|handout:4->{\After{nothing}}\begin{onlyenv}<15-16|handout:3>
      \begin{itemize}
      \item<15-> \T{type}s, \T{mode}s, and \T{storage.mode}s primarily at runtime
      \item<16-> No existing static type inference
   \end{itemize}
\end{onlyenv}
   \item<5-> Slicing algorithm \onslide<20-|handout:5->{\After{algorithm}}\begin{onlyenv}<18-19|handout:4>
      \begin{itemize}
      \item<18-> Basic slicing algorithm by \citeauthor{weiser_program_1984}~\cite{weiser_program_1984}% TODO:  inter-procedural
      \item<19-> Slicing with data-flow is relatively simple
   \end{itemize}
\end{onlyenv}
\end{enumerate}
\begin{tikzpicture}[overlay,remember picture]
   \node[below left=3mm] at(current page.north east) {\resizebox{.4\linewidth}!{\usebox\FinalSlicingStage}};
\only<8-10|handout:1>{
   \onslide<10->{\node[above left=3mm,yshift=\btdmfootheight] (@) at(current page.south east) {\scalebox{.625}{\usebox\ParseTree}};}
   \onslide<9->{\node[above,align=center,xshift=-3mm,font=\footnotesize] (@2) at (@.north) {\bR{parse(text="x <<- 4**3")}};}
   \node[above right,yshift=-2mm,align=center,font=\footnotesize] at (@2.north west) {\bR{parse(text="x <- 4^3")}};
}
\only<18-19|handout:4>{
   \node[FCite] at(current page.south west) {\cite{weiser_program_1984}: \citeauthor{weiser_program_1984}, \citetitle{weiser_program_1984} (\citeyear{weiser_program_1984})};
}
\end{tikzpicture}%
\note[itemize]{%
\item TODO
\item TODO: Otherwise: Nothing: no-one supoprts assigns etc.
}(1)%
\end{frame}

\begin{frame}
% TODO: LOGOS, TODO match flowr color and add as well
TODO: other related work together where it is useful
TODO: better color?
\end{frame}

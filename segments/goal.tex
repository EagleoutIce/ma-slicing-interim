
\section{Goal}
\subsection{Program Slicing}
\newsavebox\FinalSlicingStage
\begin{frame}[fragile,label=goal]{The Goal}
\hypertarget<1>{@Goal}{}\vspace*{-7.5mm}\lstfs{9}
\null\hfill\begin{lrbox}\FinalSlicingStage
\begin{tikzpicture}
\begin{uncoverenv}<2->
\node (@) at (0,0) {%
\begin{minted}[deletekeywords={sum,prod},escapeinside={/*}{*/}]{R}
sum  <- 0
prod <- 1
n    <- 10

for (i in 1:(n-1)) {
   sum  <- sum + i
   prod <- prod * i
}

cat("Sum:", /*\T{\only<3->{\textbf}{sum}}*/, "\n")
cat("Product:", prod, "\n")
\end{minted}
};
\end{uncoverenv}

\begin{uncoverenv}<4->
   \node[right=2.5cm] (@2) at (@.east) {%
\solsetmintedstyle{plain}
\AnimateCode{onslide={o8:{7,5,4,3,1}}, first slide=4,handout={4/1}}
\begin{minted}[deletekeywords={sum,prod},escapeinside={/*}{*/}]{R}
sum  <- 0
prod <- 1
n    <- 10

for (i in 1:(n-1)) {
   sum  <- sum + i
   prod <- prod * i
}

cat("Sum:", sum, "\n")
cat("Product:", prod, "\n")
\end{minted}
\endAnimateCode
   };
\end{uncoverenv}
\onslide<3->{
   \draw[-Kite,line cap=round] ([xshift=5mm]@.east) to[edge node={node[above] {\footnotesize\T{slice(\solGet{linenumbers}{\footnotesize10}, \textbf{sum})}}}] ([xshift=-5mm]@2.west);
}
\end{tikzpicture}\end{lrbox}\usebox\FinalSlicingStage\hfill\null
\global\setbox\FinalSlicingStage=\box\FinalSlicingStage
\note[itemize]{%
\item Wir haben R Code
\item Was er genau macht nicht wichtig
\item **Wichtige Variablen** sum & prod
\item Wollen wissen warum **sum** in **Zeile 10** den Wert hat, den es hat
\item **Prod unwichtig**
}(1)%
\end{frame}

\newsavebox\ParseTree

\begin{lrbox}{\ParseTree}
\begin{forest}
   for tree={font=\ttfamily,align=center,l sep=0pt,l sep-=5mm}
   [exprlist
      [expr
         [expr
            [\Content{SYMBOL}{x}]
         ]
         [\Content{LEFT\_ASSIGN}{{\LeftArrow/\DoubleLeftArrow}}]
         [expr
            [expr
               [\Content{NUM\_CONST}{4}]
            ]
            [\Content{\smash{\textasciicircum}}{{\textasciicircum/**}}]
            [expr
               [\Content{NUM\_CONST}{3}]
            ]
         ]
      ]
   ]
\end{forest}
\end{lrbox}

\def\After#1{\quad\textcolor{gray}{\scriptsize#1}}
\subsection{Program Slicing Requirements}
\begin{frame}[c,fragile]{Requirements}%  on other slicing algorithms
\vspace*{2mm}\begin{enumerate}
   \itemsep12pt
   \item<2-> Control-flow information (AST) \onslide<11-|handout:2->{\After{partially}}\begin{onlyenv}<6-10|handout:1>
         \begin{itemize}
         \item<6-> R provides a \T{parse} function to parse R code
         \item<7-> But the produced AST is inconsistent
      \end{itemize}
   \end{onlyenv}
   \item<3-> Data-flow information \onslide<16-|handout:3->{\After{nothing}}\begin{onlyenv}<12-15|handout:2>
      \begin{itemize}
      \item<12-> There is \T{CodeDepends}\textsuperscript{\color{gray}\cite{lang_codedepends_2018}} which does not differentiate bodies
      \item<13-> Otherwise: No existing data-flow analysis.
   \end{itemize}
\end{onlyenv}
   \item<4-> Type information \onslide<19-|handout:4->{\After{nothing}}\begin{onlyenv}<17-18|handout:3>
      \begin{itemize}
      \item<17-> \T{type}s, \T{mode}s, and \T{storage.mode}s primarily at runtime
      \item<18-> No existing static type inference
   \end{itemize}
\end{onlyenv}
   \item<5-> Slicing algorithm \onslide<22-|handout:5->{\After{algorithm}}\begin{onlyenv}<20-21|handout:4>
      \begin{itemize}
      \item<20-> Basic slicing algorithm by \citeauthor{weiser_program_1984}\textsuperscript{\color{gray}\cite{weiser_program_1984}}% TODO:  inter-procedural
      \item<21-> Slicing with data-flow is relatively simple
   \end{itemize}
\end{onlyenv}
\end{enumerate}
\begin{tikzpicture}[overlay,remember picture]
   \node[below left=3mm] at(current page.north east) {\resizebox{.4\linewidth}!{\usebox\FinalSlicingStage}};
\only<8-10|handout:1>{
   \onslide<10->{\node[above left=3mm,yshift=\btdmfootheight] (@) at(current page.south east) {\scalebox{.625}{\usebox\ParseTree}};}
   \onslide<9->{\node[above,align=center,xshift=-3mm,font=\footnotesize] (@2) at (@.north) {\bR{parse(text="x <<- 4**3")}};}
   \node[above right,yshift=-2mm,align=center,font=\footnotesize] at (@2.north west) {\bR{parse(text="x <- 4^3")}};
}
\begin{onlyenv}<12-15|handout:2>
   \node[FCite] at(current page.south west) {\cite{lang_codedepends_2018} \citeauthor{lang_codedepends_2018}, \citetitle{lang_codedepends_2018} (\citeyear{lang_codedepends_2018})};
   \onslide<15->{%
      \node[above left=5mm,yshift=\btdmfootheight,font=\footnotesize] (@) at(current page.south east) {\bR{evalq(a <- 1, envir=x)}};
   }
   \onslide<14->{%
      \node[above right,yshift=-1mm,font=\footnotesize] at(@.north west) {\bR{assign("a", 1)}};
   }
\end{onlyenv}
\only<20-21|handout:4>{
   \node[FCite] at(current page.south west) {\cite{weiser_program_1984} \citeauthor{weiser_program_1984}, \citetitle{weiser_program_1984} (\citeyear{weiser_program_1984})};
}
\end{tikzpicture}%
\note[itemize]{%
\item **CFlow**/AST
\item **Datenfluss**: Variablen
\item **Typ**-Informationen um besonders zu behandeln
\item **Slicing**-Algorithmus
\item
\item AST Inkonsistent
\item Datenfluss: hart, assign & evalq (function-reassignments, ...)
\item Typ-Informationen: Schwierig/Runtime
\item Slicing-Algorithmus: Mark **Weiser** [1984 | proposed: 1979] fehlt nur Adaption auf konkreten DF-Graph/R [existiert aber noch nicht]
\item [**Dynamic**: Bogdan Korel & Janusz Laski, erstmal ausführen | Paketversionen | tracen]
\item Other Slicers like JavaSlicer, Slicer4J, Slicer Generators wie Espresso (Java), AdaSlicer, |  Formals: Chisel...
\item  DFlow Data Flow C/C++
}(3)%
\end{frame}

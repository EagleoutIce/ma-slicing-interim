\appendix

\begin{frame}{Overview of Backup-Material}
   \tableofcontents
\end{frame}

\oldsection{Quote}
% https://rstudio-pubs-static.s3.amazonaws.com/32755_ed1dfe49c84e42c89a1a27f57a6e6557.html#/5
% https://statmodeling.stat.columbia.edu/2010/09/13/ross_ihaka_to_r/
% R for visualization
\begin{frame}[c,plain]{}
\begin{tikzpicture}[@O]
   \node (@) at(current page.center) {\begin{minipage}{.8\linewidth}
      \hyphenpenalty=10000\relax
      \Large\linespread{1.05}\only<3->{\color{gray}}\onslide<2->{\strut I have been worried for some time that R isn’t going to provide the base that we’re going to need for statistical computation in the future.\strut}%\medskip\par
   \end{minipage}};
   \onslide<3->{\node[below left,align=right] at(@.south east) {Ross Ihaka\\Co-Creator of R};}
\end{tikzpicture}
\note[itemize]{%
\item Arbeitet seit 2010 an einem nicht angekündigten aber immer wieder erwähnten Nachfolger
}(1)%
\end{frame}

\oldsection{Statistics}
\begin{frame}{Used Packages}
   \begin{center}
      \begin{columns}[t]
         \column{.42\linewidth}
            \strut In UseR scripts:\smallskip
            \begin{enumerate}
               \itemsep=0pt
               \item<3-> \link{https://cran.r-project.org/web/packages/ggplot2/index.html}{ggplot2}\I{plotting}
               \item<3-> \link{https://dplyr.tidyverse.org/}{dplyr}\I{data manipulation}
               \item<3-> \link{https://cran.r-project.org/web/packages/tidyverse/}{tidyverse}\I{packages for data science}
               \item<3-> \link{https://cran.r-project.org/web/packages/lme4/index.html}{lme4}\I{mixed-effect models}
               \item<3-> \link{https://cran.r-project.org/web/packages/plyr/index.html}{plyr}\I{more data manipulation}
            \end{enumerate}
         \column{.42\linewidth}
         \strut In package code:
         \begin{enumerate}
            \itemsep=0pt
            \item<4-> \link{https://www.rdocumentation.org/packages/stats/versions/3.6.2}{stats}\I{statisticsal functions}% is file, ...
            \item<4-> \link{https://cran.r-project.org/web/packages/R.utils/index.html}{utils}\I{basic programming functions}% is file, ...
            \item<4-> \link{https://cran.r-project.org/web/packages/rlang/index.html}{rlang}\I{working with types}
            \item<4-> \link{https://cran.r-project.org/web/packages/withr/index.html}{withr}\I{temproarily modify global state}
            \item<4-> \link{https://cran.r-project.org/web/packages/testthat/index.html}{testthat}\I{testing framework}
         \end{enumerate}
      \end{columns}
   \end{center}
   \note[itemize]{
      \item lme4 macht linear und generalized mit dem S4 Klassensystem
   }(1)
\end{frame}

\oldsection{Languageserver}
\begin{frame}[fragile]{Definition-Retrieval}
\solsetmintedstyle{plain}\lstfs{8}%
\begin{minted}[prebreak={}]{R}
paste(
   "(*|descendant-or-self::exprlist/*)[self::FUNCTION or self::OP-LAMBDA]/following-sibling::SYMBOL_FORMALS[text() = '{token_quote}' and @line1 <= {row}]",
   "(*|descendant-or-self::exprlist/*)[LEFT_ASSIGN[preceding-sibling::expr[count(*)=1]/SYMBOL[text() = '{token_quote}' and @line1 <= {row}] and following-sibling::expr[@start > {start} or @end < {end}]]]",
   "(*|descendant-or-self::exprlist/*)[RIGHT_ASSIGN[following-sibling::expr[count(*)=1]/SYMBOL[text() = '{token_quote}' and @line1 <= {row}] and preceding-sibling::expr[@start > {start} or @end < {end}]]]",
   "(*|descendant-or-self::exprlist/*)[EQ_ASSIGN[preceding-sibling::expr[count(*)=1]/SYMBOL[text() = '{token_quote}' and @line1 <= {row}] and following-sibling::expr[@start > {start} or @end < {end}]]]",
   "forcond/SYMBOL[text() = '{token_quote}' and @line1 <= {row}]",
sep = "|")
\end{minted}
\end{frame}

\oldsection{More Dataflow}
\def\G#1{\only<3->{\textsubscript{\color{gray}#1}}}
\newcommand<>\OpOn{\tikzset{@/.style={}}\only#1{\tikzset{@/.style={opacity=.5}}}}
\begin{frame}[fragile]{Example Dataflow}
\solsetmintedstyle{plain}%
\qquad\begin{uncoverenv}<2->
\begin{minipage}{.25\linewidth}
\CodeAnimationsNoFadeOut
\AnimateCode{onslide={1,2,3,4,6,0}, first slide=4,handout={4/1}}
\begin{minted}[deletekeywords={c},escapeinside=||,lineskip=4pt]{R}
a|\G0| <- 3
a|\G1| <- x|\G0| * m|\G0|

if(m|\G1| > 3) {
  a|\G2| <- 5
}

b|\G0| <- a|\G3| + c|\G0|
\end{minted}\hfill
\endAnimateCode
\end{minipage}
\end{uncoverenv}
\begin{uncoverenv}<4->
\begin{minipage}{.6\linewidth}
\centering\def\XSep{1cm}
\def\YSep{8mm}
\resizebox*!{6cm}{\begin{tikzpicture}[comm/.style={rectangle,draw,text width=9mm,align=center,minimum height=5mm,font=\ttfamily,fill=lightgray!25!btdm@background},d/.style={comm,rounded corners=2pt},u/.style={comm,rounded rectangle},T/.style={font=\scriptsize,gray}]
   \OpOn<5-8>
   \onslide<4->{\scope[@]\node[d] (a0) at (0,0) {a\G0};\endscope}
   \OpOn<6-8>
   \onslide<5->{\scope[@]\node[d,below=\YSep] (a1) at (a0.south) {a\G1};
   \draw[densely dotted] (a0) to[edge node={node[left,T] {same (def-def)}}] (a1);
   \node[u,right=\XSep] (x0) at (a1.east) {x\G0};
   \node[u,right=\XSep] (m0) at (x0.east) {m\G0};
   \draw[-Kite] (a1) to[edge node={node[below,T] {def-by}}] (x0);
   \draw[-Kite] (a1) to[bend left=15,edge node={node[above,T] {def-by}}] (m0);
   \endscope}

   \OpOn<7-8>
   \onslide<6->{
   \scope[@]
   \node[u,below=\YSep] (m1) at (a1.south) {m\G1};
   \draw[densely dotted] (m0) to[edge node={node[below,sloped,T] {same (read-read)}}] (m1);
   \endscope
   }
   \OpOn<8-8>
   \onslide<7->{
      \scope[@]
      \node[d,below=\YSep] (a2) at (m1.south) {a\G2};
      \draw[densely dotted] (a1.210) to[bend right,edge node={node[left,T] {same (def-def)}}] (a2.150);
      \endscope
   }

   \onslide<8->{
   \node[d,below=\YSep] (b0) at (a2.south) {b\G0};
   \node[u,right=\XSep] (a3) at (b0.east) {a\G3};
   \node[u,right=\XSep] (c0) at (a3.east) {c\G0};
   \draw[-Kite] (b0) to[edge node={node[above,T] {def-by}}] (a3);
   \draw[-Kite] (b0) to[bend right=15,edge node={node[below,T] {def-by}}] (c0);

   \node[above=\YSep,comm,text width=0pt,minimum size=1mm,shape=diamond] (@1) at (a3.north) {};
   \draw[-Kite] (a3) to[edge node={node[right,T] {read (may)}}] (@1);
   \draw[densely dotted,-Kite] (@1) -- (a2);
   \draw[densely dotted,-Kite] (@1) -- (a1);
   }
\end{tikzpicture}}
\end{minipage}
\end{uncoverenv}
\end{frame}

\section{R Fun}
\begin{frame}[fragile,c]{Modifying Environments and Functions}
\solsetmintedstyle{plain}\lstfs{10}%
\begin{minted}[escapeinside={/*}{*/},aboveskip=0pt]{R}
x <- new.env()
evalq(a <- 1, envir=x)
evalq(a, envir=x)




f <- function(x) { y <- x * 3; y }
body(f)[[3]] <- quote(x)
f(2) # 2
\end{minted}
\end{frame}

\begin{frame}[fragile,c]{Modifying Assignments}
\solsetmintedstyle{plain}\lstfs{10}%
\begin{minted}[escapeinside={/*}{*/}]{R}
/*\onslide<2->*/f <- function(x) { body(f)[[2]] <<- 3 }
/*\onslide<2->*/f(2)/*\onslide<3->*/ # <invisible>
/*\onslide<3->*/f(2)/*\onslide<4->*/ # 3



/*\onslide<5->*/f <- function(x) a + b
/*\onslide<5->*/f(2) /*\onslide<6->*/# Error in f(2) : object 'a' not found
/*\onslide<6->*/formals(f) <- alist(a=,b=40)
/*\onslide<6->*/f(2) /*\onslide<7->*/# 42/*\onslide<1->*/
\end{minted}
\end{frame}